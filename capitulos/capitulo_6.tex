\chapter{Características da Realidade Virtual}

\section{Imersiva}

Realidade imersiva, significa tal como o nome diz, uma realidade em que o seu utilizador sente-se imerso. É uma sensação em que o seu utilizador experimenta sensações quase reais dentro do ambiente virtual, podendo este interagir com os seus elementos. 
Os exemplos mais comuns da realidade imersiva são por exemplo simuladores de voo, capacetes, oculus rift, luvas virtuais, fato virtual, Icaros \cite{RV_inter}. 

\section{Não imersiva}

Realidade não imersiva, é basicamente o oposto do que foi dito na realidade imersiva em que o seu usuário tem a sensação de que está num ambiente real. A realidade não imersiva não consiste na sensação de inclusão experimentada pelo usuário, pois ele não se sente dentro do ambiente virtual, é esta realidade que a maior parte das pessoas estão acostumadas a experimentar, por exemplo, quando um usuário está a jogar jogos num computador comum, ou seja, esteja a visualizar o jogo no ecrã e a interagir nele com o rato e teclado, ou até mesmo um comando, a isto é que definido como sendo a realidade não imersiva \cite{RV_inter}.

\section{Interativa}
Já foi dito o que significava realidade imersiva, mas o que significa a realidade interativa? 
A realidade interativa é aquela em que o seu usuário pode interagir com os elementos dentro da realidade virtual, por exemplo, um usuário que esteja num ambiente simulado pode manipular objetos virtuais e o equipamento mais comum para que isso seja possível é o uso de luvas virtuais \cite{RV_inter}.







