\chapter{Vantagens e Desvantagens}

\section{Vantagens}

A Realidade Virtual  veio facilitar em grande parte processos outrora não dominados pelo ser humano, reduzindo imenso os custos assim como a perda de vidas humanas. 
Através da simulação podemos controlar diversos equipamentos à distância. 
Possibilita a execução de várias simulações de forma a treinar profissionais sem qualquer tipo de perigo real, com baixos custos de manutenção. No caso de simuladores de voo por exemplo as emissões de gases para a atmosfera não existem, o que contribui para redução do flagelo do efeito de estufa.
Na medicina a realidade virtual também pode ajudar muito mais eficazmente no tratamento de fobias, treino de operações, etc.


\section{Desvantagens}

Uma vez que esta tecnologia está a dar os seus primeiros passos na história da humanidade, a sua produção ainda acarreta custos elevados, como por exemplo o seu custo. O preço de um automóvel novo é equivalente ao custo dos equipamentos da Realidade Virtual, mas atenção não estamos a falar de um simples equipamento como o Oculus Rift, mas sim, da realidade totalmente imersiva. 

Tal como acima mencionado a Realidade Virtual é uma tecnologia recente e posto isso ainda não está desenvolvida de maneira que possa ser executada num tradicional computador pessoal pois a capacidade do seu processamento ainda é muito elevada o que requer poderosos computadores.

E como o perigo certamente não é uma vantagem, vale a pena numerar alguns perigos desta nova tecnologia. Um dos perigos é quando os usuários estiverem a utilizar por exemplo os Oculus Rift certamente não terão noção do que se passa à sua volta, isso tornará o usuário completamente vulnerável ao mundo real, se este estiver em locais públicos poderá facilmente ser roubado ou algo pior, se estiver dentro de casa também existirá o perigo pois o usuário não verá os objetos que existem á sua volta no mundo real, estará submerso num mundo onde os objetos reais não existem e isso poderá levar a certos danos físicos se no caso estiver a explorar o mundo virtual.




